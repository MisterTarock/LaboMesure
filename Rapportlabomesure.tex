\documentclass[11pt,a4paper]{report}
\usepackage[utf8]{inputenc}
\usepackage[french]{babel}
\usepackage[T1]{fontenc}
\usepackage{amsmath}
\usepackage{amsfonts}
\usepackage{amssymb}
\usepackage{graphicx}
\usepackage[left=2cm,right=2cm,top=2cm,bottom=2cm]{geometry}
\begin{document}
\title{Rapport laboratoire de mesure}


\chapter{Définition}	
\subsection{But}
\subsection{Hypothèse}
\subsubsection{Formule1}
\subsubsection{Formule2}
\subsection{Schema fonctionnel}

\chapter{Liste du matériel}
\begin{itemize}
\item 8 jauges de déformations (type FCA-5-23)
\item un corps d'épreuve (diabolo en aluminium)
\item 5 charges experimentales de masse connue
\item un générateur de courant continu (5V)
\item Matériel d'acquisition (Ordinateur avec software dédié)
\item une Balance à levier
\end{itemize}	

\chapter{Procédure expérimentale}

(mettre le dessin)

\begin{enumerate}
\item Réquilibrer le pont de Wheatsone
\item Center le corps d'épreuve sous la bille qui permet d'appliquer la charge.
\item Deposer la charge sur le plateau.
\item Relever la différence de potentiel affichée à l'écran
\item Relever la valeur de la masse affchée
\item Repeter les opétration 2 à 5 en augmentant la valeur de la charge par pas de 500g en veillant à ne pas déposer une charge de 5kg
\item Repeter la procédure pour des centrage de l'application de la masse différente
\end{enumerate}
Il est important de rester délicat avec la matériel lors du changement de masse ou du changement de centrage d'application de la force, Sous peine de faussé les mesures suivantes
\chapter{Mesures \& observations}
\section{Graphique}
\section{Tableau de valeurs}

	
\chapter{Interpretation}	
\chapter{Conclusion}	
\chapter{Annexe}	



\end{document}